\documentclass[10pt,a4paper]{article} % oder: {book}, {report}
\usepackage[ngerman]{babel} % Neue deutsche Silbentrennung
\usepackage[utf8]{inputenc} % Zeichenkodierung UTF-8 (Unicode). Editor muss
\usepackage[T1]{fontenc} % Korrektes Trennen von Wörtern mit Umlauten
% und Akzenten.
\usepackage{pgfplots}
\pgfplotsset{width=10cm, compat=1.15}
\usepackage{lmodern} % Type-1-Schriftart für das PDF-Dokument
\begin{document}
\section{Laufzeitverhalten abschätzen}
\subsection{Summe der geraden Fibonacci Zahlen}
     \begin{tabular}{|c|c|}\hline
       \textbf{max int}  & \textbf{Zeit in Nanosek.} \\ \hline
       5000           & 149             \\ \hline
       10000       & 521             \\ \hline
       50000        & 12129             \\ \hline
       100000        & 48143             \\ \hline
       500000        & 80218             \\ \hline
     \end{tabular}

    \vspace{0.7cm}

\begin{tikzpicture}
  \begin{axis}[
  title={Aufgabe 3: Summe der geraden Fibonacci Zahlen},
    axis lines=middle,
    axis line style={->},
    x label style={at={(axis description cs:0.5,-0.1)},anchor=north},
    y label style={at={(axis description cs:-0.2,.5)},rotate=90,anchor=south},
    xlabel={Anzahl der getesteten Zahlen},
    ylabel={benötigte Zeit in Nanosekunden}]

    \addplot coordinates {
      (0, 0)
      (5000, 149)
      (10000, 521)
      (50000, 12129)
      (100000, 48143)
      (500000, 80218)
    };
  \end{axis}
\end{tikzpicture}
\end{document}
